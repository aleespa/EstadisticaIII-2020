\documentclass[aspectratio=169,spanish]{beamer}
\usepackage[utf8]{inputenc}
\title{Estadística III}
\subtitle{Pruebas de 1 y 2 muestras}
\author{Alejandro López Hernández}
\institute{FES Acatlán - UNAM}
\date{\today}
\usetheme{Pittsburgh}
\usecolortheme{beaver}

\begin{document}

\frame{\titlepage}
\begin{frame}
\frametitle{Índice}
\tableofcontents
\end{frame}

\begin{frame}
\frametitle{Pruebas de 1 muestra}
\section{Pruebas de 1 muestra}
Las pruebas de 1 una muestra consiste en tener una población a la cual queremos comparar en dos instacias distintas, por ejemplo, cuando se aplica cierto tratamiento y se quiere medir el impacto de tal tratamiento. Asumiremos que tenemos $2n$ datos tales que $Z_i=X_i-Y_i$ son independientes, tambien asumiremos que $Z_i$ tienen la misma distribución la cual tiene media $\theta$, este parmetro será el de nuestro interes. 
\end{frame}


\begin{frame}
\frametitle{Prueba de Wilcoxon - Rangos signados}
\section{Prueba de Wilcoxon - Rangos signados}
Está prueba sirve para probar la hipotesis $H_0:\theta =0$, para probar esa hipotesis se costruye el estadístico $T^{+}$ con los rangos de los valores de $|Z_i|$, se define el estadístico como $$ T^{+}=\sum_{i=1}^{n}R_i\varphi_i$$
donde $\varphi_i$ vale 1 cuando $|Z_i|>0$ y 0 cuando $|Z_i|<0$.
\end{frame}
\begin{frame}
\frametitle{Prueba de Wilcoxon - Rangos signados}
Se pueden realizar 3 pruebas:
\begin{itemize}
\item Prueba de la cola superior, $H_0:\theta=0$ vs $H_1:\theta>0$, $H_0$ se debe rechazar con una significancia de $\alpha$ si $T^{+}\ge t_\alpha$.
\item Prueba de la cola inferior, $H_0:\theta=0$ vs $H_1:\theta<0$, $H_0$ se debe rechazar con una significancia de $\alpha$ si $T^{+}\le \frac{n(n+1)}{2} -t_\alpha$.
\item Prueba de dos colas, $H_0:\theta=0$ vs $H_1:\theta=0$, $H_0$ se debe rechazar con una significancia de $\alpha$ si $T^{+}\ge t_{\alpha/2}$ o $T^{+}\le \frac{n(n+1)}{2} -t_{\alpha/2}$
\end{itemize}
\end{frame}

\begin{frame}
\frametitle{Prueba de Wilcoxon - Rangos signados}
Se puede realizar una modificación en el estadístico para poder comparlo contra los cuantiles de la normal, esto debido a que $\mathbb{E}_0(T^{+})=\frac{n(n+1)}{4}$ y que $\text{Var}(T^{+})_0=\frac{n(n+1)(2n+1)}{12}$, por lo tanto podemos construir el esadístico 
$$T^{*}=\frac{T^{+}-\frac{n(n+1)}{4}}{\left(\frac{n(n+1)(2n+1)}{12}\right)^{1/2}}$$
\end{frame}

\begin{frame}
Con $T^{*}$ las pruebas quedan como:
\begin{itemize}
\item Prueba de la cola superior, $H_0:\theta=0$ vs $H_1:\theta>0$, $H_0$ se debe rechazar con una significancia de $\alpha$ si $T^{*}\ge z_\alpha$.
\item Prueba de la cola inferior, $H_0:\theta=0$ vs $H_1:\theta<0$, $H_0$ se debe rechazar con una significancia de $\alpha$ si $T^{*}\le -t_\alpha$.
\item Prueba de dos colas, $H_0:\theta=0$ vs $H_1:\theta=0$, $H_0$ se debe rechazar con una significancia de $\alpha$ si $|T^{*}|\ge z_{\alpha/2}$ 
\end{itemize}
\end{frame}


\begin{frame}
\frametitle{Pruebas de 2 muestras}
\section{Pruebas de 2 muestras}
Con las pruebas de 2 muestras, nuestro interes será comparar un par de muestras que provienen de poblaciones con alguna caracteristica que las diferencian, y queremos investigar en si esa característica es significativa. Otra forma de verlo es con el caso particular en el que tenemos una poblacion a la cual le aplicamos cierto tratamiento, entonces queremos ver si el tratamiento tuvo un efecto \emph{positivo} o \emph{negativo}.
\end{frame}


\begin{frame}
\frametitle{Prueba de Mann-Whitney-Wilcoxon}
\section{Prueba de Mann-Whitney-Wilcoxon}
Esta prueba consiste en probar la hipotesis de que ambas muestras tienen la misma distribución, es decir $H_0:F(t)=G(t)$, sin embargo en ningún momento se especifica cual es la distribución de $F$ ó $G$, por lo tanto podemos cambiar la prueba a $H_0:\Delta =\mathbb{E}(X)-\mathbb{E}(Y)=0$.
La forma de construir el estadístico es calcular los rangos conjuntos de $X$ y $Y$, con sus $N=m+n$, si $S_j$ denota los rangos de $Y$, entonces el estadístico se calcula como:
$$W=\sum_{j=1}^{n}S_j$$
\end{frame}


\begin{frame}
\frametitle{Prueba de Mann-Whitney-Wilcoxon}
Se pueden realizar 3 pruebas:
\begin{itemize}
\item Prueba de la cola superior, $H_0:\Delta=0$ vs $H_1:\Delta>0$, $H_0$ se debe rechazar con una significancia de $\alpha$ si $W\ge\omega_\alpha$.
\item Prueba de la cola inferior, $H_0:\Delta=0$ vs $H_1:\Delta<0$, $H_0$ se debe rechazar con una significancia de $\alpha$ si $W\le n(m+n+1)-\omega_\alpha$.
\item Prueba de dos colas, $H_0:\Delta=0$ vs $H_1:\Delta=0$, $H_0$ se debe rechazar con una significancia de $\alpha$ si $W\ge\omega_{\alpha/2}$ o $W\le n(m+n+1)-\omega_{\alpha/2}$.
\end{itemize}
\end{frame}


\begin{frame}
\frametitle{Prueba de Mann-Whitney-Wilcoxon}
En el caso de que tengamos una gran cantidad de datos $N$, podemos calcular un estadístico alterno que se pueda comparar con una normal, eso gracias al hecho de que $\mathbb{E}_0(W)=\frac{n(m+n+1)}{2}$ y que $\text{Var}_0(W)=\frac{mn(m+n+1)}{12}$, por lo tanto podemos construir el estadístico:
$$W^{*}=\frac{W-(n(m+n+1)/2)}{(mn(m+n+1)/12)^{1/2}}$$
\end{frame}
\begin{frame}
\frametitle{Prueba de Mann-Whitney-Wilcoxon}
De esta forma las pruebas se convierten en:
\begin{itemize}
\item Prueba de la cola superior, $H_0:\Delta=0$ vs $H_1:\Delta>0$, $H_0$ se debe rechazar con una significancia de $\alpha$ si $W^{*}\ge z_\alpha$.
\item Prueba de la cola inferior, $H_0:\Delta=0$ vs $H_1:\Delta<0$, $H_0$ se debe rechazar con una significancia de $\alpha$ si $W^{*}\le -z_\alpha$
\item Prueba de dos colas, $H_0:\Delta=0$ vs $H_1:\Delta=0$, $H_0$ se debe rechazar con una significancia de $\alpha$ si $|W^{*}|\ge z_{\alpha/2}$
\end{itemize}
\end{frame}

\end{document}

