\documentclass[aspectratio=169,spanish]{beamer}
\usepackage[utf8]{inputenc}
\title{Estadística III}
\subtitle{Pruebas de bondad de ajuste}
\author{Alejandro López Hernández}
\institute{FES Acatlán - UNAM}
\date{\today}
\usetheme{Pittsburgh}
\usecolortheme{beaver}

\begin{document}

\frame{\titlepage}



\begin{frame}
\frametitle{Índice}
\tableofcontents
\end{frame}



\begin{frame}
\frametitle{Introducción}
\section{Introducción}
La idea de las pruebas de bondad de ajuste es comparar la función de distribución de nuestros datos ($\hat{F}_n$) con una función de distribución dada ($F_0$). Nuestro objetivo será encontar estadístos que nos ayuden a aceptar o rechazar la siguiente prueba: $$H_0: F=F_0$$
\end{frame}

\begin{frame}
\frametitle{Introducción}
Ejemplo: Supongamos que tenemos los datos $$X =0.254, 1.23,4.566,2.165,1.23,1.829,5,3.23$$
Una pregunta interesante es si los datos tiene una distribución \textit{uniforme}, para probar la hipótesis, es necesario calcular $\hat{F}_n$ y con ella la prueba sería de la forma:
$$H_0: F=F_0\sim U(0,5)$$
\end{frame}
\begin{frame}
\frametitle{Introducción}
Algunas alternativas para medir las diferencias en las distribucciones son:
\begin{itemize}
\item $D_n^{+}=\sup_{-\infty<t<\infty}(\hat{F}_n(t)-F_0(t))$
\item $D_n^{-}=\sup_{-\infty<t<\infty}(F_0(t)-\hat{F}_n(t))$
\item $D_n=\sup_{-\infty<t<\infty}|F_0(t)-\hat{F}_n(t)|=\max(D_n^{+},D_n^{-})$
\item $V_n =D_n^{+}+D_n^{-}$
\item $C_n=\int(F_0(t)-\hat{F}_n(t))^2dF_0(t)$
\item $A_n=\int\frac{(\hat{F}_n(t)-F_0(t))^2}{F_0(t)(1-F_0(t))}dF_0(t)$
\item $w_{n,k,g}=\int(F_0(t)-\hat{F}_n(t))^k g(F_0(t))dF_0(t)$
\item $w_{n,k,g}=\int(F_0(t)-\hat{F}_n(t))^k g(F_0(t))dF_0(t)$
\end{itemize}
\end{frame}

\begin{frame}
\frametitle{Prueba de Kolgomorov-Smirnov}
\section{Prueba de Kolgomorov-Smirnov}
La prueba de Kolgomorov-Smirnov se define como $$D_n=\sup_{-\infty<t<\infty}|F_0(t)-\hat{F}_n(t)|$$ 
Se define de esa manera debido al teorema de Gilvenko-Cantelli, que nos dice que $\sup_{-\infty<t<\infty}|F(t)-\hat{F}_n(t)|\rightarrow 0\quad \text{a.s}$, es decir que la máxima distancia entre la distribución empírica y la real tiende a 0. Por lo tanto si nuestra $F_0$ es la distribución real, se espera que $D_n$ sea pequeño.
\end{frame}


\begin{frame}
\frametitle{Prueba de Kolgomorov-Smirnov}
Para calcular $D_n$ solo es necesario conocer las observaciones $X_1,X_2,...,X_n$, si $X_{\{i\}}$ es el i-ésimo estadístico de orden, se puede probar que $$D_n =\max_{1\le i\le n}max(\frac{i}{n}-F_0(X_{\{i\}}),F_0(X_{\{i\}})-\frac{i-1}{n})$$
\end{frame}
\begin{frame}
\frametitle{Prueba de Kolgomorov-Smirnov}
Con el siguiente resultado encontraremos la distribución asintótica de $D_n$
\begin{block}{Teorema 1}
Sea $X_1,X_2,...,X_n\sim F_0$ y sea $D_n=\sup_{t}|F_0(t)-\hat{F}_n(t)|$ entonces, suponiendo que $F_0$ es continua; $$\sqrt{n}D_n \rightarrow \sup_{0\le t\le 1}|B(t)|\quad \text{en distribución}$$
Donde $B(t)$ es un puente Browniano.
\end{block}
\end{frame}

\begin{frame}
\frametitle{Prueba de Kolgomorov-Smirnov}
Afortunadamente, se puede desarrollar la distribución del puente browniano. 
\begin{block}{Proposición 1}
Bajo la hipótesis $H_0$ entonces 
$$\lim_{n}\mathbb{P}(\sqrt{n}D_n \le \lambda) =1-2\sum_{j=1}^{\infty}(-1)^{j-1}e^{-2j^2\lambda^2}$$
\end{block}
Con está distribución podemos calcular cuantiles y derivar regiones de rechazo.
\end{frame}


\begin{frame}
\frametitle{Prueba de Cramér-von Mises}
\section{Prueba de Cramér-von Mises}
La idea de este estadístico es medir el area que separa $\hat{F}_n$ de $F_0$, el estadístico se define como:
$$C_n=\int(F_0(t)-\hat{F}_n(t))^2dF_0(t)$$
\end{frame}
\begin{frame}
\frametitle{Prueba de Cramér-von Mises}
De forma analoga a el estadístico $D_n$, $C_n$ se puede escribir en función de los estadísticos de orden: 
$$C_n=\frac{1}{12n}+\sum_{i=1}^{n}\left(F_0(X_{\{i\}})-\frac{2i-1}{2n}\right)^2$$ 
\end{frame}
\begin{frame}
\frametitle{Prueba de Cramér-von Mises}
Con el siguiente resultado encontraremos la distribución asintótica de $C_n$
\begin{block}{Teorema 2}
Sea $X_1,X_2,...,X_n\sim F_0$ y sea $C_n=\int(F_0(t)-\hat{F}_n(t))^2dF_0(t)$ entonces, suponiendo que $F_0$ es continua; $$nC_n \rightarrow \int_0^1B^2(t)dt\quad \text{en distribución}$$
Donde $B(t)$ es un puente Browniano.
\end{block}
\end{frame}

\begin{frame}
\frametitle{Prueba de Cramér-von Mises}
De forma similar que con $D_n$ se puede calcular la distribución asintótica de $C_n$
\begin{block}{Proposición 2}
Bajo la hipótesis $H_0$ entonces 
$$ \lim_{n}\mathbb{P}(n C_n > \lambda)=\frac{1}{\pi}\sum_{j=1}^{\infty}(-1)^{j+1}\int_{(2j-1)^2\pi^2}^{4j^2\pi^2}\sqrt{\frac{-\sqrt{y}}{\sin(\sqrt{y})}}\frac{e^{-\frac{xy}{2}}}{y}dy$$
\end{block}
De igual forma podemos crear regiones de rechazo con esta distribución.
\end{frame}






\begin{frame}
\frametitle{Prueba de Anderson-Darling}
\section{Prueba de Anderson-Darling}
La idea es tambien medir el área en que separa $\hat{F}_n$ de $F_0$, sin embargo, el termino $F_0(t)(1-F_0(t))$ busca tener una mayor ponderación en la región donde la distribución $F_0(t)$ tiene mayor incertidumbre. El estadistico se define como 
$$A_n=\int\frac{(\hat{F}_n(t)-F_0(t))^2}{F_0(t)(1-F_0(t))}dF_0(t)$$
\end{frame}
\begin{frame}
\frametitle{Prueba de Anderson-Darling}
De forma analoga a los estadísticos anteriores $A_n$ se puede escribir como:
$$A_n=-n-\frac{1}{n}\left[\sum_{i=1}^{n}(2i-1)(\log F_0(X_{\{i\}})+\log(1-F_0(X_{\{n-i+1\}}))\right]$$
\end{frame}
\begin{frame}
\frametitle{Prueba de Anderson-Darling}
Con el siguiente resultado encontraremos la distribución asintótica de $A_n$
\begin{block}{Teorema 3}
Sea $X_1,X_2,...,X_n\sim F_0$ y sea $A_n=\int\frac{(\hat{F}_n(t)-F_0(t))^2}{F_0(t)(1-F_0(t))}dF_0(t)$ entonces, suponiendo que $F_0$ es continua; $$nA_n \rightarrow \int_0^1\frac{B^2(t)}{t(1-t)}dt\quad \text{en distribución}$$
Donde $B(t)$ es un puente Browniano.
\end{block}
\end{frame}

\begin{frame}
\frametitle{Pruebas de 2 Muestras}
\section{Pruebas de 2 Muestras}
Supongamos que tenemos dos muestras independientes, $X_1,X_2,...,X_n\sim F_0$ y $Y_1,Y_2,...,Y_m\sim F$, entonces para comparar las 2 muestras hacemos la prueba de hipótesis $H_0: F=F_0$, para construir la prueba usamos $F_m$, $G_m$ la función de distribución empírica de $X_i$ y $Y_i$ respectivamente. De forma análoga a lo desarrollado para las prebas de una muestra, necesitamos una forma de comparar ambas distribuciones. Para ello usamos el estadístico: 
$$D_{n,m} = \sup_{0\le t\le 1}|F_n(t)-G_m(t)|$$

\end{frame}

\begin{frame}
\frametitle{Pruebas de 2 Muestras}
La distribución límite de $D_{n,m}$ es la siguiente:
\begin{block}{Teorema 4}
Sea $X_1,X_2,...,X_n\sim F_0$ y $Y_1,Y_2,...,Y_m\sim F$ donde $F_0$ y $F$ son distribuciones continuas. Bajo la hipotesis $H_0: F=F_0$:
$$ \lim_{n,m} \mathbb{P}(\sqrt{\frac{nm}{n+m}}D_{n,m} \le \lambda) = 1-2\sum_{j=1}^{\infty}(-1)^{j-1}e^{-2j^2\lambda^2}$$
dado que para algún $0<\gamma <1$, $\frac{m}{n+m}\rightarrow \gamma$
\end{block}
\end{frame}
\end{document}