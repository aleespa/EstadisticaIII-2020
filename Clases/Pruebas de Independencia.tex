\documentclass[aspectratio=169,spanish]{beamer}
\usepackage[utf8]{inputenc}
\title{Estadística III}
\subtitle{Pruebas de independencia}
\author{Alejandro López Hernández}
\institute{FES Acatlán - UNAM}
\date{\today}
\usetheme{Pittsburgh}
\usecolortheme{beaver}

\begin{document}

\frame{\titlepage}



\begin{frame}
\frametitle{Índice}
\tableofcontents
\end{frame}



\begin{frame}
\frametitle{Introducción}
\section{Introducción}
El problema que intentaremos resolver es cuando tenemos una muestra bivariada y queremos saber la relación que existe entre las dos variables aleatorias, en particular la indepenecia. Nuestro supuesto es que tenemos una muestra bivariada independiente de $n$ datos, del tipo $(X_1,Y_2),...,(X_n,Y_n)$. La hipótesis que queremos probar es de la forma $$H_0:[F_{XY}(x,y)=F_{X}(x)F_{Y}(y)\text{ para cualquier par }(x,y)]$$
\end{frame}


\begin{frame}
\frametitle{Prueba de Kendall}
\section{Prueba de Kendall}
La prueba de Kendall se basa en la cantidad $\tau = 2\mathbb{P}((Y_2-Y_1)(X_2-X_1)>1)-1$, está cantidad se propone debido a que si $X$ fuera independiente de $Y$, $\tau$ deberia de ser $0$, por lo tanto nuestra prueba de hipotesis la probaremos buscando valores pequeños de $\tau$, sin embargo debemos probar todas las combinaciones de pares entre las observaciones.  
\end{frame}

\begin{frame}
\frametitle{Prueba de Kendall}
Para el cálculo del estadístico, utilizamos la siguiente función 
\begin{equation*}
Q((a,b),(c,d)) = \left\lbrace
\begin{array}{ll}
1 & \text{si } (d-b)(c-a)>0\\
-1 & \text{si } (d-b)(c-a)<0
\end{array}
\right.
\end{equation*}
El estadístico de Kendall se define como: 
$$K = \sum_{i=1}^{n-1}\sum_{j=i+1}^{n}Q((X_i,Y_i),(X_j,Y_j)) $$
\end{frame}

\begin{frame}
\frametitle{Prueba de Kendall}
Para calcular la distribución de $K$, se puede aproximar con la distrubición normal, para eso utilizamos el hecho de que $\mathbb{E}(K)=0$ y $\text{Var}(K)=\frac{n(n-1)(2n+5)}{18}$¸ con ello podemos modificar el estadístico como:
$$K^*=\frac{K}{(n(n-1)(2n+5)/18)^{1/2}}$$
\end{frame}

\begin{frame}
\frametitle{Prueba de Spearman}
\section{Prueba de Spearman}
La prueba de Spearman, no resulta ser tan intuitiva, se define como la correlación de los rangos de los datos, es decir:
$$ r_s = \frac{12\sum_{i=1}^{n}[R_i -\frac{n+1}{2}][S_i -\frac{n+1}{2}]}{n(n^2-1)}$$
de igual forma se busca que $ r_s$ sea una cantidad baja cuando la hipótesis sea cierta.
\end{frame}


\begin{frame}
\frametitle{Prueba de Spearman}
De igual forma se puede aproximar cuando se tiene una gran cantidad de datos por una normal, tenemos que $\mathbb{E}(r_s)=0$ y $\text{Var}(r_s)=\frac{1}{n-1}$, por lo tanto podemos modificar el estadístico como:
$$ r_s^* = \sqrt{n-1}r_s$$
Y las regiones de rechazo las podemos poner en terminos de la distribución normal.
\end{frame}

\end{document}